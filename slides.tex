\documentclass{beamer}
\begin{document}
\title{Compiler Development in Industry}
\author{Aditya Kumar and Sebastian Pop}
\institute{SARC: Samsung Austin Research Center}
\date{March 31, 2015}
%\date{\today}

\frame{\titlepage}

\frame{\frametitle{What sorts of things does a compiler engineer do?}
  day-to-day, working on:
  \begin{itemize}
  \item optimizations
  \item implementing language extensions
  \item improving the breadth of support for hardware features
  \item working with several teams (from different labs)
  \item in open source, collaborate with different companies
  \item retargetting a compiler to company's new processor
  \item FP/int arithmetics, static analysis, machine learning \ldots{}
  \item mostly working on state-of-the-art systems and technologies
  \item fixing bugs and keeping systems running
  \item reading and writing research papers
  \end{itemize}
}

\frame{\frametitle{Why compilers are interesting}
  \begin{itemize}
  \item compiling programs
  \item making program execution faster
  \item shrinking executable size
  \item detecting security issues
  \item program analysis and transformation
  \item obfuscation
  \end{itemize}
}

\frame{\frametitle{What drives compiler developments?}
  \begin{itemize}
  \item microprocessor and architecture changes
    \begin{itemize}
    \item changes to the ISA: add new instructions, intrinsics
    \item memory and pipeline model: latency of instructions
    \end{itemize}
  \item benchmarks (SPEC, EEMBC, Geekbench, PARSEC, \ldots{})
    \begin{itemize}
    \item avoid regressions: correctness, performance
    \item code is fixed, compiler changes
    \item performance analysis: compare perf of different compilers
    \item profiling critical path, analyzing assembler, perf counters
    \end{itemize}
  \item systems (LAMP stack, Android, Hadoop, \ldots{})
    \begin{itemize}
    \item program and compiler can change
    \item why fixing the compiler when source code can be improved?
    \end{itemize}
  \end{itemize}
}

\frame{\frametitle{Performance analysis and bug reports}
  on the hot path:
  \begin{itemize}
  \item how to increase IPC, ILP, use of SIMD, parallelism, \ldots{}
  \item how to decrease memory transactions, latency, \ldots{}
  \item why existing optimizations do not apply?
  \item print the IR and decisions of each pass
  \item testcase reduction: delta, c-reduce, bugpoint (LLVM)
  \item analyze the deficiencies in the optimization
  \item share testcase with the community to gather ideas on fix
  \item implement the best idea
  \end{itemize}
}

\frame{\frametitle{Continuous testing}
  100+ commits per day
  \begin{itemize}
  \item compiler changes impact the shape of IR
  \item pattern matching optimizations may not trigger
  \end{itemize}

  How do we keep track of the impact of all these changes?
  \begin{itemize}
  \item nightly testsuite runs
  \item only variable: revisions of the compiler
  \item perf and correctness regression detection
  \item noise, smoothing, statistics, derivatives, etc.
  \item git bisect is awesome
  \item community report as a bug
  \end{itemize}
}

\frame{\frametitle{Open source compiler releases}
  \begin{itemize}
  \item GCC
    \begin{itemize}
    \item released once a year, quality based release (no P1 bugs)
    \item dev stages: new development, bug fixing, regression only
    \item release branches only accept regression fixes backports
    \item released compilers are supported for several years
    \end{itemize}
  \item LLVM
    \begin{itemize}
    \item released twice a year, deadline based release
    \item always in new development mode
    \item once released, few or no patches backported
    \item no community support for released compilers
    \end{itemize}
  \end{itemize}
}

\frame{\frametitle{Compiler engineering perspectives}
  \begin{itemize}
  \item number of compiler engineering positions are increasing
  \item ARM ecosystem adds more processor variety and new architectures on the market
  \item many HW companies have compiler teams
  \item compiler teams are as central to the technology performance as microprocessor architecture teams
  \end{itemize}
}

\frame{\frametitle{Internship in the SARC compiler group}
  
}

\end{document}
